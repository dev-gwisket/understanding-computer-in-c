\section{Compiled vs. Interpreted Programming Languages}

Computers can only understand machine codes. Thus, source codes must be
translated into machine codes. Programming languages are also be able to
be classified into three categories by when is the language translated.

Firstly, a \emph{compiled language} translates the source codes
\emph{before} a program is executed. Thus, it generates executables or
binary files which are formed with machine codes. Programs which generate
them are called \emph{compilers}. Examples of programming languages of
this type are C and C++.

Secondly, a \emph{interpreted language} translates the source codes
line-by-line \emph{immediately} when a program is executed. Thus, it does
not generates anything. Programs which read and execute commands are
called \emph{interpreters}. Actually, there are almost no programming
languages which are purely interpreted languages.

Finally, a language which is compiled and interpreted is compiled into
byte codes which are similar to machine codes but machine-independent.
Then, a interpreter reads the byte codes and executes commands.
Programming languages like Python or Java are often called interpreted
languages, however, in a strict sense, they are not interpred languages
but compiled and interpreted languages.
